% Created 2024-05-28 Tue 16:49
% Intended LaTeX compiler: pdflatex
\documentclass[11pt]{article}
\usepackage[utf8]{inputenc}
\usepackage[T1]{fontenc}
\usepackage{graphicx}
\usepackage{longtable}
\usepackage{wrapfig}
\usepackage{rotating}
\usepackage[normalem]{ulem}
\usepackage{amsmath}
\usepackage{amssymb}
\usepackage{capt-of}
\usepackage{hyperref}
\author{Tobi Lehman}
\date{\today}
\title{Kolmogorov-Arnold Representation Theorem with Haskell code examples}
\hypersetup{
 pdfauthor={Tobi Lehman},
 pdftitle={Kolmogorov-Arnold Representation Theorem with Haskell code examples},
 pdfkeywords={},
 pdfsubject={},
 pdfcreator={Emacs 29.3 (Org mode 9.6.15)}, 
 pdflang={English}}
\begin{document}

\maketitle
\tableofcontents


\section{Introduction}
\label{sec:org6c4be2c}
The Kolmogorov-Arnold representation theorem states that any continuous function \(f:[0,1]^n \to \mathbb{R}\) of several variables can be represented as a finite sum of continuous functions \(\phi_{ij}\) of one variable.

\section{Theorem Statement}
\label{sec:org82dd4e8}

$$ f(x_1, x_2, \ldots, x_n) = \sum_{i=1}^{2n+1} g_i \left( \sum_{j=1}^{n} \varphi_{ij}(x_j) \right)$$

\section{Proof}
\label{sec:org1d43ae2}


\section{Haskell Implementation}
\label{sec:org5fe754d}

\begin{verbatim}
module KolmogorovArnold where

-- Example function
f :: Double -> Double -> Double
f x y = x^2 + y^2

-- Functions g_i and phi_ij for the representation
g1, g2 :: Double -> Double
g1 z = z
g2 z = z

phi1, phi2 :: Double -> Double
phi1 x = x
phi2 y = y

-- Kolmogorov-Arnold representation
karRepresentation :: Double -> Double -> Double
karRepresentation x y = g1 (phi1 x + phi2 y) + g2 (phi1 x + phi2 y)
\end{verbatim}
\end{document}
