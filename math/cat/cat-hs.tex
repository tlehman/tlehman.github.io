% Created 2024-09-08 Sun 15:34
% Intended LaTeX compiler: pdflatex
\documentclass[11pt]{article}
\usepackage[utf8]{inputenc}
\usepackage[T1]{fontenc}
\usepackage{graphicx}
\usepackage{longtable}
\usepackage{wrapfig}
\usepackage{rotating}
\usepackage[normalem]{ulem}
\usepackage{amsmath}
\usepackage{amssymb}
\usepackage{capt-of}
\usepackage{hyperref}
\author{Tobi Lehman}
\date{\today}
\title{Category Theory in Haskell}
\hypersetup{
 pdfauthor={Tobi Lehman},
 pdftitle={Category Theory in Haskell},
 pdfkeywords={},
 pdfsubject={},
 pdfcreator={Emacs 29.4 (Org mode 9.6.15)}, 
 pdflang={English}}
\begin{document}

\maketitle
\tableofcontents


\section{Introduction}
\label{sec:org356bc08}

Category Theory is the mathematical study of other mathematical theories.
I first really appreciated it while studying topology. Topology is hard.
\href{../grp/grp-hs.html}{Group Theory} is easier. A lot of topology is about simplifying topology problems
by reducing them to group theory problems.

\subsection{Motivating Example}
\label{sec:org535e7af}

Suppose you have a continuous function \(f : X \to Y\) where
\(X\) and \(Y\) are topological spaces. You want to know if \(X\) is
homeomorphic to \(Y\) (topology-speak for "the same").

However, proving a homeomorphism can be really hard, since the spaces \(X\) and \(Y\)
could have a uncountable (\href{https://tobilehman.com/archive/tlehman.blog/p/transfinite-numbers.html}{transfinite}) number of points. The shape might be really weird.
So topologists developed a way to compute a group called the fundamental group of a
space. The fundamental group \(\pi_1(X)\) has elements which are loops in the space \(X\),
the group operation is just following one loop and then following the other loop.

There's a very important theorem we proved about the fundamental group. Before we state it,
let's look at this diagram:

\begin{verbatim}
\begin{CD}
  X @>{f}>> Y \\
  @V{\pi_1(-)}VV @V{\pi_1(-)}VV \\
  \Pi_1(X) @>>> \Pi_1(Y)
\end{CD}
\end{verbatim}
\end{document}
